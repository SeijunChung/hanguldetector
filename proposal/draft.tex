\documentclass[%
 reprint,
%showpacs,preprintnumbers,
nofootinbib,
nobibnotes,
%bibnotes,
amsmath,amssymb,
aps,
]{revtex4-1}
\usepackage{graphicx}% Include figure files
\usepackage{dcolumn}% Align table columns on decimal point
\usepackage{bm}% bold math
\usepackage{enumitem}
\usepackage{listings}
\usepackage{titlesec}
\usepackage{etoolbox}
\usepackage{flushend}
\usepackage{balance}

\makeatletter
\patchcmd{\ttlh@hang}{\parindent\z@}{\parindent\z@\leavevmode}{}{}
\patchcmd{\ttlh@hang}{\noindent}{}{}{}
\makeatother

\lstset{frame=tb,
aboveskip=3mm,
belowskip=3mm,
tabsize=2
}

\def\bibsection{\section*{\refname}}
\renewcommand\thesection{\arabic{section}}
\renewcommand\thesubsection{\thesection.\arabic{subsection}}
\begin{document}
\titlespacing*{\subsection}{0pt}{1.1\baselineskip}{\baselineskip}

\title{RECOGNIZATION AND RECOMMENDATION OF HANGUL FONTS USING COMPUTER VISION BASED
TECHNIQUES}% Force line breaks with \\

\author{CHUNG SEIJUN 2014-23341 OUM ?????? KIM  ??????? CHOO 2014-18868   }
\affiliation{%
 Computer Vision 2019-1\\Department of Computer Science\\Seoul National University
}%
\date{April 26,2019}

\begin{abstract}
  Fonts are a set of pre-designed ... I don't know what should go in to an
  abstract. PLEASE FILL ME IN
\begin{description}
\item[Objective]
This project recognizes different Hangul fonts based on Computer Vision
    Techniques and provide akin fonts that avoids legal issues.
\end{description}
\end{abstract}

\maketitle

%\tableofcontents

\section{\label{sec:level1}INTRODUCTION}
From the monospaced fonts on a VT100 terminal to the
Seoul fonts by the Seoul city government, the range of available fonts to users have expanded, due to the rise of tools for font creation.

Fonts are predefined collections of letters, varying in design and size.
Different fonts are utilized based on the purpose of the user, as an
intermediary of the user's purposes and emotions.

However, as more and more fonts are being introduced to the market, the quest
for the appropriate font is either an evergoing battle with vast images,
or an effortless defeat, known as the default font. Even when one has a
desired font with its image in hand, the search is analogous to finding a needle in the
haystack. Also, finding the appropriate font is not just a matter of choice, as it can spur legal
disputes as well.\cite{lawsuit}

As of now, there is no way to search the desired font from an input of image one has found in books,
magazines, or from a street sign;

Compared to Hangul fonts, English fonts consist of 26 upper and 26 lower case
letters, a total of 52 characters. The limited number of characters, and their
simple form makes English font recognition not as challenging and previous
researches have prodcued high recognition rate(add reference?),

However, the Hangul font consists of 2,350(KS X 1001) or 11,172(KPS 9566)
characters, each composed of onset, nucelus, and codas. Due to this complicated
format, research on font recogntion of Hangul characters is not active, compared to other Latin alphabets.
In this project, we will use KS X 1001 character set.


\section{\label{sec:level1}PAPER SURVEY}

\subsection{\label{sec:level2}Classical Computer Vision}
We shall utilize many techniques learned from our class and materials.

\subsection{\label{sec:level2}Optical Font Recognition Using Typographical
Featurs}
This paper\cite{zramdini} aims to identify typeface, weight, slope and the size of
text from an image block without any knowledge of the content of the text.

\subsection{\label{sec:level2}Large-Scale Visual Font Recognition}
This paper\cite{chen} addresses the large-scale visual font recognition (VFR)
problem, which aims at automatic identification of the typeface.

\section{\label{sec:level2}KEY FEATURES}
\subsection{\label{sec:level3} Separating Hangul character from image}
We cannot expect input as an image of a single character.
Input image may include many characters, so we should separate it
character by character using xx technique.

\subsection{\label{sec:level3} Feature Detection}
A Study on Typology for Hangul Fonts Identification and Classification in
terms of character's typeface, character size and character slope and etc.
based on Computer Vision Techniques.

\subsection{\label{sec:level3} Machine Learning(Optional)}
Rather than conservative computer theories, we can use machine learning techiniques
in detecting characters or fonts.
Mass of Hangul character fonts recogntion with CNN, Hidden Markov Model with
reference to paper\cite{tensmeyer} or other materials.

\subsection{\label{sec:level3} Estimating Surface Normal Vector of Hangul
Fonts (Optional)}
??

\pagebreak
\section{\label{sec:level3} ALGORITHM}

\begin{lstlisting}
separate characters from input image
for each c in characters:
	find idx such that c == ref_font[idx]
	for each font in fonts:
		compare(c, font[idx])
	select most similar font
select font with most concensus
		
		
	
\end{lstlisting}


\section{\label{sec:level2}EXPECTED RESULTS}
We aim for the following goals:
\begin{enumerate}[topsep=0pt, itemsep=-1ex, partopsep=1ex, parsep=1ex]
  \item Extract distinct features of Korean characters based on Computer Vision
    Techniques
  \item Recognize/Recommend the provided font

\end{enumerate}

\section{\label{sec:level2}DATASET}
Since there's not enough data set existing, we will make new dataset.
We will download .ttf font files offered by Naver Corp.
Using the downloaded font files, we will generate 64 * 64 image file for 2,350 Korean letters.
Therefore, there will be 2,350 * (\# of distinct fonts) image files of size 64 * 64.
%The experimental data set used in this paper is
%\textbf{\textit{PHD08}}\cite{ham}, a large-sacle Korean character
%database. \textbf{\textit{PHD08}} has 2,187 samples for each of 2,350 Korean letters, consisting
%of a total 5,139,450 samples.

\begin{thebibliography}{1}
\bibitem{lawsuit} The Font Bureau, Inc. v. NBC Universal, Inc. et al\\
https://dockets.justia.com/docket/new-york/nyedce/1:2009cv04286/296847
\bibitem{zramdini} Abdelwahab Zramdini, et al. {\em Optical Font Recognition
  Using Typographical Features}.\/ IEEE TRANSACTIONS ON PATTERN ANALYSIS AND
    MACHINE INTELLIGENCE, 1988.
\bibitem{chen} Guang Chen, et al. {\em Large-Scale Visual Font Recognition}.
\/ CVPR, 2014.
\bibitem{tensmeyer} Chris Tensmeyer, et al. {\em Convolution Neural Networks
  for Font Classification}. CVPR, 2017.
\bibitem{ham} D. Ham, D.Lee, I.Jung and I. Oh. {\em Construction of Printd
  Hangul Character Database PHD08} Journal of the Korea Contents Association,
    vol.8, no.11, pp. 33-40, Nov. 2008.
\end{thebibliography}




\flushend
\end{document}
%
% ****** End of file apssamp.tex ******
